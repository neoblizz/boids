The core idea of boids is put forward by Craig W. Reynolds in 1986~\cite{Reynolds:1987}. There is no need to implement a real commander which sends commands to each boid for maintaining the flock. Each individual only need to follow the 3 principles: alignment, cohesion, and separation.

To simulate the aggregate motion, for each boid, we iterate and collect both position and velocity(both speed and direction) information of all other flockmates to calculate average velocity, average distance to other flockmates, and central position of this flock. 

\subsection{Alignment}
The average velocity is used for alignment. The average velocity is a 3D vector with two parameters average heading and average speed. Each boid adjusts its heading towards the average heading of the whole flock so that the whole flock is always heading in the same direction. It also checks its speed with average speed. If the current boid is too fast, it slows down waiting for other flockmates catching up, otherwise speeds up. The more important part of the alignment is the heading match process. We notice that the speed alignment does not need to be specifically implemented since the separation process would be used to adjust speed instead. We set a minimum and a maximum speed for the flock and use Clamp function to adjust the speed. Heading is more important because it directly results in whether it’s jerky or smooth when the flock makes a turn. To achieve this, the following equation is implemented:
\vspace{3mm}
\begin{center}
$v_j = (\frac{1}{n}\sum_{i=1,i\neq{j}}^{n}{v_i})\cdot S + v_j \cdot (1-S)$
\end{center}
\vspace{3mm}
Where $v_j$ is the velocity of current boid, $n$ is the number of boids in the flock. $S$ is a parameter set to control the boid motion. Its value varies from 0 to 1. A larger $S$ means the changing vector takes more advantage than the current status so that the flock is more tending to change. The default setting of $S$ is 0.5. Thus, the changing vector is simplified to $v_{average} - v_j$. This change will attach to each boid by adding a force in the same direction as this changing vector. A weight value $w_a$ is set to determine the force strength. The final force add to the boid would be like this in our project implementation:
\vspace{3mm}
\begin{center}
$f_j = (v_{avg} - v_j)_{magnitude} \cdot w_a$
\end{center}
\vspace{3mm}
The whole process mentioned above is processed every frame to all boids.
\subsection{Cohesion}
Flock center position is used for cohesion. A force to the center of the flock is always added to each boid. Note that when flock crosses an obstacle, the boids may be divided into 2 or more parts then concentrate together again after then. In this case, there is no need to consider the center of each cluster when the flock is temporarily divided. Otherwise, the obstacle process of this may look like one flock initiated divide into two or more instead of an obstacle come across a single flock. The cohesion equation is:
\vspace{3mm}
\begin{center}
$v_j = ((\frac{1}{n}\sum_{i=1,i\neq{j}}^{n}{pos_i})-pos_j)\cdot S \cdot w_c + v_j \cdot (1-S)$
\end{center}
\vspace{3mm}
Here $pos$ is the position of a boid, $w_c$ is the weight of cohesion. We could see that the further boid[j] away from the center of the flock, a larger force will be implemented.

\subsection{Separation}
Average boid distance is used for separation. We set a collision test for each boid in the flock to make them separate. Each boid has a private zone and we check whether other flockmates entered this zone. If so, a separation force is added each frame until leaving. The equation is like the following:
\vspace{3mm}
\begin{center}
$v_j = (\frac{1}{n}\sum_{i=1,i\neq{j}}^{n}{\frac{v_i}{dist(pos_i,pos_j)}})\cdot S \cdot w_s + v_j \cdot (1-S)$
\end{center}
\vspace{3mm}
$dist(pos_i,pos_j)$ is distance vector from boid[i] to boid[j]. The further boid[i] away from boid[j] the less influence will be added for separation. The private zone mentioned above is set to avoid unnecessary calculation. We consider both the position and velocity of other flockmates in this process. We notice that if only the position of two boids is considered, the flock could not be separated well in some situations (e.g. When boids are too close to each other at the beginning). This is also used for the speed matching process that we mentioned in the alignment section. A separation weight parameter $w_s$ is also set for separation force.\break
%\vspace{5mm}
\\
The $S$ parameter from 3 equations have the same function and set to be the same in our implementation but it does not necessary to be the same. The 3 weight value are all set to be 1, it is does not make a significant change to the output when the value changes. We believe they are not necessary parameters. Recall the algorithm mentioned in this section is only implemented to each boid and no other general control system is added. This is built to stick to the natural flock condition.