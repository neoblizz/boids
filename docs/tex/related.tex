Simulating a crowed boids has a long history in computer animation. At the SIGGRAPH conference in 1985, Susan Amakraut, Michael GirardSusan Amakraut, and Michael Girard from Ohio State University revealed a small piece of animation "Motion studies for a work in progress entitled 'Eurythmy'". \cite{1985}In their animation, several birds flying from a birdcage, crossing several cylinder obstacles and then landing on the ground. Each bird has its winds waving when flying and makes sure they don't hit each other. They call their work Eurythmy Force System which is a 3D vector based acceleration force that allows an object to move from one point to another. There is also another repellent force that avoids collision of flockmates and obstacles. The force is linear and developers only need to define the force and computers can simulate all complicate motions based on that.
\\
Craig Reynolds' work has been extended in several ways since 1986. The basic model had rules added on top of that, which mapped things like emotions between objects~\cite{Delgado:2007}, these emotions could be an attraction factor or fear. An introduction of "change of leadership" was presented by Hartman and Benes\cite{Hartman:2006}.\\
The Media Lab of MIT also has several related motion animation research. But they are not building the system by group control but distribute models instead.\cite{MIT} Another interesting example similar to this is a popular game software called 'Creature', revealed in 1996\cite{c_wiki}. By 1999, there are more than 500,000 games been sold. The game developer Cliff D. and Grand S. make an interesting discussion about why this game is so popular and foreseeing a future of game animation and AI in their paper. The strategy of 'Creature' is maintaining a networking system in each agent(artificial object), and the agents are influenced via this network with each other under an inheritance algorithm\cite{c_article}.\\
Particle System also has similarity to crow simulation. Both two are describing a large scale number of objects' movements and each element is moving unpredictably. Particle System focuses more on phenomenons like fire, smoke, cloud, sand, etc. They are all combined with a large number of small particles but each has its unique identity. The particle can form an overall look of the system by changing its color, position, transparency, speed, etc. But the boids system is more complex than this since particle could not have a complicate motion like an animal. Also, quantum does not have a shape and individual motion. So it could be defined as a 2.5D system. Although some software like MAYA and XSI \cite{maya}could further implement a physical object, these designed objects still unable to form a further change like shape distortion.\\
To be more specific, in MAYA, collisions could not be implemented between particles, so only force field model could be implemented to keep a distance between each particle. With the development of this system, more approaches like AI are also implemented for perfection. For example in 'MUMMY RETURNS'\cite{MR}, an AI particle simulation is designed. Several core ideas could be referred and adapted to crowed simulation as well.